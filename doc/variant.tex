\subsection{Решение с линейными ограничениями}

В общем случае у нас 9 неизвестных и 6 уравнений.
Чтобы решить эту систему необходимо ввести дополнительные взаимосвязи между измерениями векторов $\vec{S}_0$, $\vec{S}_1$, $\vec{S}_2$.
Отсутсвие ограничений для вектора реакции значает, что точка, к которой он приложен, зафиксирована.
Введение взаимосвязи между измерениями вектора реакции означает, что у точки есть степень свободы вдоль которой нет реакции опоры.

Пусть $\vec b_1$ - направляющий вектор степени свободы точки $p_1$.
Тогда реакция $\vec S_1$ должна быть перпендикулярна $\vec b_1$:
\begin{equation}
	\vec b_1 \cdot \vec S_1 = 0.
\end{equation}

Если задать для одной точки две степени свободы, это будет означать, что она может двигаться в плоскости, заданной пересекающимися прямыми, а реакция опоры должна быть перпендикулярна этой плоскости.

Пусть $\vec b_{21}$ и $\vec b_{22}$ - направляющие степеней свободы точки $p_2$;

\begin{equation}
\left\{
	\begin{gathered}
		\vec b_{21} \cdot \vec S_2 = 0;\\
		\vec b_{22} \cdot \vec S_2 = 0.
	\end{gathered}
\right.
\end{equation}

Нормаль к плоскости можно найти как $$\vec n_2 = \vec b_{21} \times \vec b_{22}$$

Генерация векторов $\vec b_{21}$ и $\vec b_{22}$ по заданному $\vec n_2$ возможна, но в статье не рассматривается.

Таким образом, при одной степени свободы у $p_1$ и двух у $p_2$, система принимает вид:

\begin{equation}
\left\{
	\begin{gathered}
		\vec{S}_0 + \vec{S}_1 + \vec{S}_2 = -\vec{F},\\
		\vec{p}_0 \times \vec{S}_0 + \vec{p}_1 \times \vec{S}_1 + \vec{p}_2 \times \vec{S}_2 = -\vec{M},\\
		\vec b_1 \cdot \vec S_1 = 0,\\
		\vec b_{21} \cdot \vec S_2 = 0,\\
		\vec b_{22} \cdot \vec S_2 = 0.
	\end{gathered}
\right.
\end{equation}
То же в проекциях
\begin{equation}
\left\{
	\begin{gathered}
		S_{0,x} + S_{1,x} + S_{2,x} = -F_x,\\
		S_{0,y} + S_{1,y} + S_{2,y} = -F_y,\\
		S_{0,z} + S_{1,z} + S_{2,z} = -F_z,\\
		p_{0,y}S_{0,z} - p_{0,z} S_{0,y} +	p_{1,y}S_{1,z} - p_{1,z} S_{1,y} + p_{2,y}S_{2,z} - p_{2,z} S_{2,y} = - M_x;\\
		p_{0,z}S_{0,x} - p_{0,x} S_{0,z} +	p_{1,z}S_{1,x} - p_{1,x} S_{1,z} + p_{2,z}S_{2,x} - p_{2,x} S_{2,z} = - M_y;\\
		p_{0,x}S_{0,y} - p_{0,y} S_{0,x} +	p_{1,x}S_{1,y} - p_{1,y} S_{1,x} + p_{2,x}S_{2,y} - p_{2,y} S_{2,x} = - M_z;\\
		b_{1,x}S_{1,x} + b_{1,y}S_{1,y} + b_{1,z}S_{1,z} = 0;\\
		b_{21,x}S_{2,x} + b_{21,y}S_{2,y} + b_{21,z}S_{2,z} = 0;\\
		b_{22,x}S_{2,x} + b_{22,y}S_{2,y} + b_{22,z}S_{2,z} = 0.
	\end{gathered}
\right.
\end{equation}

Составим матрицу
\begin{gather}
	A = \begin{pmatrix}
		1 & 0 & 0 & 1 & 0 & 0 & 1 & 0 & 0 \\
		0 & 1 & 0 & 0 & 1 & 0 & 0 & 1 & 0 \\
		0 & 0 & 1 & 0 & 0 & 1 & 0 & 0 & 1 \\
		0 & -p_{0,z} & p_{0,y} & 0 & -p_{1,z} & p_{1,y} & 0 & -p_{2,z} & p_{2,y} \\
		p_{0,z} & 0 & - p_{0,x} & p_{1,z} & 0 & - p_{1,x} &	p_{2,z} & 0 & - p_{2,x} \\
		-p_{0,y} & p_{0,x} & 0 & -p_{1,y} & p_{1,x} & 0 & -p_{2,y} & p_{2,x} & 0 \\
		0 & 0 & 0 & b_{1,x} & b_{1,y} & b_{1,z} & 0 & 0 & 0 \\
		0 & 0 & 0 & 0 & 0 & 0 & b_{21,x} & b_{21,y} & b_{21,z} \\
		0 & 0 & 0 & 0 & 0 & 0 & b_{22,x} & b_{22,y} & b_{22,z}
	\end{pmatrix};\\
	B = \begin{pmatrix}
		-F_x\\
		-F_y\\
		-F_z\\
		-M_x\\
		-M_y\\
		-M_z\\
		0 \\ 0 \\ 0
	\end{pmatrix};\;
	X = \begin{pmatrix}
		S_{0,x} \\ 
		S_{0,y} \\ 
		S_{0,z} \\ 
		S_{1,x} \\ 
		S_{1,y} \\ 
		S_{1,z} \\ 
		S_{2,x} \\
		S_{2,y} \\
		S_{2,z}
	\end{pmatrix}.
\end{gather}

Матрицу $A$ можно представить как составную
\begin{gather}
	A = \begin{pmatrix}
		E_3 & \vrule & E_3 & \vrule & E_3  \\ \hline
		\hat{p}_0 & \vrule & \hat{p}_1 & \vrule & \hat{p}_2 \\ \hline
		0 & \vrule & \vec b_1^T & \vrule & 0 \\
		0 & \vrule & 0  & \vrule & \vec b_{21}^T \\
		0 & \vrule & 0  & \vrule & \vec b_{22}^T
	\end{pmatrix};\\
\end{gather}
Здесь каждая ограниченная область содержит матрицу $3\times 3$.

Вектора $B$ и $X$ тоже можно рассматривать как составные
\begin{gather}
	B = \begin{pmatrix}
		-\vec F \\
		-\vec M \\
		\vec 0
	\end{pmatrix};\;
	X = \begin{pmatrix}
		\vec S_0 \\
		\vec S_1 \\
		\vec S_2
	\end{pmatrix}.
\end{gather}
