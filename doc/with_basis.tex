\subsection{Рассмотрим частный случай}
Если имеется возможность рассчитать $\vec{M}$ относительно любой точки, перенесём начало системы координат в точку $p_0$.

\begin{align}
	\vec{p}_0 = \vec{0};
\end{align}

Введём для расчёта базис
\begin{equation}
	\left\{
		\begin{array}{l}
		\vec{e}_0 = \vec{p}_1 \times \vec{p}_2,\\
		\vec{e}_1 = \vec{p}_1,\\
		\vec{e}_2 = \vec{p}_2.
		\end{array}
	\right.
\end{equation}

Тогда уравнение моментов примет вид

\begin{equation}
	\sum\limits_{i=0}^2 
		\begin{vmatrix} 
		[ \vec e_1,\; \vec e_2 ] &  [ \vec e_2,\; \vec e_0 ] & [ \vec e_0,\; \vec e_1 ] \\ 
		p_{i,{e_0}} & p_{i,{e_1}} & p_{i,{e_2}} \\ 
		S_{i,{e_0}} & S_{i,{e_1}} & S_{i,{e_2}}
		\end{vmatrix}
		= -\vec{M}.
\end{equation}

Рассчитаем произведения
$$[ \vec e_1,\; \vec e_2 ] = \vec p_1 \times \vec p_2 = \vec e_0.$$
$$[ \vec e_2,\; \vec e_0 ] 
= [ \vec p_2, [ \vec p_1,\; \vec p_2 ] ] 
= \vec p_1 p_2^2 - \vec p_2 \left(\vec p_1 \cdot \vec p_2 \right).$$

$$[ \vec e_0,\; \vec e_1 ]
 = - [ \vec p_1,\; [ \vec p_1,\; \vec p_2 ] ]
 = - \vec p_1 \left(\vec p_1 \cdot \vec p_2 \right) + \vec p_2 p_1^2.$$
 
Обозначим
$$[ \vec e_2,\; \vec e_0 ] = \vec e_1';$$
$$[ \vec e_0,\; \vec e_1 ] = \vec e_2';$$

Получился ещё один базис - $\left( \vec e_0; \vec e_1'; \vec e_2' \right) $.

\begin{equation*}
	\begin{vmatrix} 
		\vec e_0 &  \vec e_1' & \vec e_2' \\ 
		p_{0,{e_0}} & p_{0,{e_1}} & p_{0,{e_2}} \\ 
		S_{0,{e_0}} & S_{0,{e_1}} & S_{0,{e_2}}
	\end{vmatrix}
	+
	\begin{vmatrix} 
		\vec e_0 &  \vec e_1' & \vec e_2' \\ 
		p_{1,{e_0}} & p_{1,{e_1}} & p_{1,{e_2}} \\ 
		S_{1,{e_0}} & S_{1,{e_1}} & S_{1,{e_2}}
	\end{vmatrix}
	+
	\begin{vmatrix} 
		\vec e_0 &  \vec e_1' & \vec e_2' \\ 
		p_{2,{e_0}} & p_{2,{e_1}} & p_{2,{e_2}} \\ 
		S_{2,{e_0}} & S_{2,{e_1}} & S_{2,{e_2}}
	\end{vmatrix}
	= -\vec{M}.
\end{equation*}
\begin{equation*}
	\begin{vmatrix} 
		\vec e_0 &  \vec e_1' & \vec e_2' \\ 
		0 & 1 & 0 \\ 
		S_{1,{e_0}} & S_{1,{e_1}} & S_{1,{e_2}}
	\end{vmatrix}
	+
	\begin{vmatrix} 
		\vec e_0 &  \vec e_1' & \vec e_2' \\ 
		0 & 0 & 1 \\ 
		S_{2,{e_0}} & S_{2,{e_1}} & S_{2,{e_2}}
	\end{vmatrix}
	= -\vec{M}.
\end{equation*}
\begin{equation*}
	\vec e_0 S_{1,{e_2}} - \vec e_2' S_{1,{e_0}} + \vec e_1'S_{2,{e_0}} - \vec e_0 S_{2,{e_1}}
	= -\vec{M}.
\end{equation*}
\begin{equation*}
	\left(S_{1,{e_2}} -  S_{2,{e_1}} \right)\vec{e_0}
	+ S_{2,{e_0}} \vec e_1' 
	- S_{1,{e_0}} \vec e_2' 
	= -\vec{M}.
\end{equation*}

\begin{equation}
\left\{\begin{array}{l}
	S_{1,{e_2}} -  S_{2,{e_1}} = - M_{e_0},\\
	S_{2,{e_0}} = -M_{e_1'},\\
	S_{1,{e_0}} = M_{e_2'}.
\end{array}\right.
\end{equation}

Уравнения сил имеет вид
\begin{equation*}
\vec{S}_0 + \vec S_1 + \vec S_2 = -\vec{F}.
\end{equation*}

Чтобы сделать задачу статически определимой, нужно сформулировать ограничения подвижности точек опоры.

\subsubsection{С неподвижной опорой}

Положим, что опора 0 неподвижна. Она лишает систему трёх степеней свободы.\\
Тогда две другие опоры должны лишать систему остальных трёх степеней свободы.
Пусть опора 1 скользит вдоль прямой, а опора 2 опирается на плоскость.


\begin{equation}
\left\{\begin{array}{l}
	\vec{b}\times\vec{S_1} = \vec{0},\\
	\vec n \cdot \vec S_2 = 0.
\end{array}\right.
\end{equation}
где $\vec{b}$ - направляющий прямой, а $\vec n$ - нормаль плоскости

\begin{multline*}
\begin{vmatrix} 
\vec e_0 &  [ \vec e_2,\; \vec e_0 ] & [ \vec e_0,\; \vec e_1 ] \\ 
b_{e_0} & b_{e_1} & b_{e_2} \\ 
S_{1,{e_0}} & S_{1,{e_1}} & S_{1,{e_2}}
\end{vmatrix} 
= \\ =
\vec e_0 \left( b_{e_1} S_{1,{e_2}} - b_{e_2} S_{1,{e_1}} \right)
+ \vec e_1' \left( b_{e_2} S_{1,{e_0}} - b_{e_0} S_{1,{e_2}} \right)
+ \vec e_2' \left( b_{e_0} S_{1,{e_1}} - b_{e_1} S_{1,{e_0}} \right);
\end{multline*}

Для расчёта скалярного произведения потребуется метрический тензор

\begin{equation}
g = 
\begin{pmatrix}
	\vec e_0 \cdot \vec e_0 & \vec e_0 \cdot \vec e_1 & \vec e_0 \cdot \vec e_2 \\
	\vec e_1 \cdot \vec e_0 & \vec e_1 \cdot \vec e_1 & \vec e_1 \cdot \vec e_2 \\
	\vec e_2 \cdot \vec e_0 & \vec e_2 \cdot \vec e_1 & \vec e_2 \cdot \vec e_2
\end{pmatrix}
=
\begin{pmatrix}
	e_0^2 & 0 & 0 \\
	0 & e_1^2 & \vec e_1 \cdot \vec e_2 \\
	0 & \vec e_2 \cdot \vec e_1 & e_2^2
\end{pmatrix}
\end{equation}

\begin{multline*}
	\vec n \cdot \vec S_2 = \sum\limits_{i=0}^{2}\sum\limits_{j=0}^{2} g_{ij} n_{e_i} S_{2,e_j} =\\=
	e_0^2 n_{e_0} S_{2,{e_0}} +
	e_1^2 n_{e_1} S_{2,{e_1}} +
	\vec e_1 \cdot \vec e_2 n_{e_2} S_{2,{e_1}} +
	\vec e_1 \cdot \vec e_2 n_{e_1} S_{2,{e_2}} +
	e_2^2 n_{e_2} S_{2,e_2}
	= \\ =
	e_0^2 n_{e_0} S_{2,{e_0}} +
	\left( e_1^2 n_{e_1} + \vec e_1 \cdot \vec e_2 n_{e_2} \right) S_{2,{e_1}} +
	\left( \vec e_1 \cdot \vec e_2 n_{e_1} + e_2^2 n_{e_2} \right) S_{2,{e_2}}.
\end{multline*}

Система примет вид
\begin{equation}
\left\{\begin{array}{l}
	e_0^2 n_{e_0} S_{2,{e_0}} +
	\left( e_1^2 n_{e_1} + \vec e_1 \cdot \vec e_2 n_{e_2} \right) S_{2,{e_1}} +
	\left( \vec e_1 \cdot \vec e_2 n_{e_1} + e_2^2 n_{e_2} \right) S_{2,{e_2}} = 0, \\

	b_{e_1} S_{1,{e_2}} - b_{e_2} S_{1,{e_1}} = 0, \\
	b_{e_2} S_{1,{e_0}} - b_{e_0} S_{1,{e_2}} = 0, \\
	b_{e_0} S_{1,{e_1}} - b_{e_1} S_{1,{e_0}} = 0, \\

	S_{0,{e_0}} + S_{1,{e_0}} + S_{2,{e_0}} = - F_{e_0},\\
	S_{0,{e_1}} + S_{1,{e_1}} + S_{2,{e_1}} = - F_{e_0},\\
	S_{0,{e_2}} + S_{1,{e_2}} + S_{2,{e_2}} = - F_{e_0},\\

	S_{1,{e_2}} -  S_{2,{e_1}} = - M_{e_0},\\
	S_{2,{e_0}} = -M_{e_1'},\\
	S_{1,{e_0}} = M_{e_2'}.
\end{array}\right.
\end{equation}

\begin{equation*}
\left\{\begin{array}{l}
	S_{1,{e_0}} = M_{e_2'},\\
	S_{2,{e_0}} = -M_{e_1'},\\
	
	- e_0^2 n_{e_0} M_{e_1'} +
	\left( e_1^2 n_{e_1} + \vec e_1 \cdot \vec e_2 n_{e_2} \right) S_{2,{e_1}} +
	\left( \vec e_1 \cdot \vec e_2 n_{e_1} + e_2^2 n_{e_2} \right) S_{2,{e_2}} = 0, \\

	b_{e_1} S_{1,{e_2}} - b_{e_2} S_{1,{e_1}} = 0, \\
	b_{e_2} M_{e_2'} - b_{e_0} S_{1,{e_2}} = 0, \\
	b_{e_0} S_{1,{e_1}} - b_{e_1} M_{e_2'} = 0, \\

	S_{0,{e_0}} + M_{e_2'} - M_{e_1'} = - F_{e_0},\\
	S_{0,{e_1}} + S_{1,{e_1}} + S_{2,{e_1}} = - F_{e_0},\\
	S_{0,{e_2}} + S_{1,{e_2}} + S_{2,{e_2}} = - F_{e_0},\\

	S_{1,{e_2}} -  S_{2,{e_1}} = - M_{e_0}.
\end{array}\right.
\end{equation*}
\begin{equation*}
\left\{\begin{array}{l}
	S_{1,{e_0}} = M_{e_2'},\\
	S_{2,{e_0}} = -M_{e_1'},\\
	
	S_{1,{e_1}} = \frac{b_{e_1}}{b_{e_0}} M_{e_2'}, \\
	S_{1,{e_2}} = \frac{b_{e_2}}{b_{e_0}} M_{e_2'}, \\
	b_{e_1} S_{1,{e_2}} - b_{e_2} S_{1,{e_1}} \equiv 0, \\

	\left( e_1^2 n_{e_1} + \vec e_1 \cdot \vec e_2 n_{e_2} \right) S_{2,{e_1}} +
	\left( \vec e_1 \cdot \vec e_2 n_{e_1} + e_2^2 n_{e_2} \right) S_{2,{e_2}} = e_0^2 n_{e_0} M_{e_1'}, \\

	S_{0,{e_0}} = - F_{e_0} - M_{e_2'} + M_{e_1'},\\
	S_{0,{e_1}} + S_{1,{e_1}} + S_{2,{e_1}} = - F_{e_0},\\
	S_{0,{e_2}} + S_{1,{e_2}} + S_{2,{e_2}} = - F_{e_0},\\

	S_{1,{e_2}} -  S_{2,{e_1}} = - M_{e_0}.
\end{array}\right.
\end{equation*}
\begin{equation*}
\left\{\begin{array}{l}
	S_{1,{e_0}} = M_{e_2'},\\
	S_{1,{e_1}} = \frac{b_{e_1}}{b_{e_0}} M_{e_2'}, \\
	S_{1,{e_2}} = \frac{b_{e_2}}{b_{e_0}} M_{e_2'}, \\
	
	S_{2,{e_0}} = -M_{e_1'},\\
	S_{2,{e_1}} = M_{e_0} + \frac{b_{e_2}}{b_{e_0}} M_{e_2'},\\
	
	S_{2,{e_2}} 
	= 
	\frac{
		e_0^2 n_{e_0} M_{e_1'}
		- \left( e_1^2 n_{e_1} + \vec e_1 \cdot \vec e_2 n_{e_2} \right) S_{2,{e_1}}
	}
	{\left( \vec e_1 \cdot \vec e_2 n_{e_1} + e_2^2 n_{e_2} \right)}
	, \\

	S_{0,{e_0}} = - F_{e_0} - M_{e_2'} + M_{e_1'},\\
	S_{0,{e_1}} = - F_{e_0} - S_{1,{e_1}} - S_{2,{e_1}},\\
	S_{0,{e_2}} = - F_{e_0} - S_{1,{e_2}} - S_{2,{e_2}},\\

\end{array}\right.
\end{equation*}

